\documentclass[11pt,letterpaper]{article}

\usepackage{amsmath}
\usepackage{multirow}

\begin{document}

\title{CS114 (Spring 2020) Written Assignment 1\\Text Classification}
\author{Due February 14, 2020}
\date{}
\maketitle

\section{Na\"ive Bayes}

The following problem is from the Jurafsky and Martin book, Exercise 4.2, reproduced below.\\

\noindent Given the following short movie reviews, each labeled with a genre, either comedy or action:

\begin{center}
\begin{tabular}{|c|c|}
\hline 
document & class \\ 
\hline 
fly fast shoot love & action \\ 
\hline 
fun couple love love & comedy \\ 
\hline 
fast furious shoot & action \\ 
\hline 
couple fly fast fun fun & comedy \\ 
\hline 
furious shoot shoot fun & action \\ 
\hline 
\end{tabular} 
\end{center}

\noindent and a new document D:\\

fast couple shoot fly\\

\noindent compute the most likely class for D. Assume a naive Bayes classifier and use add-1 smoothing for the likelihoods.\\

\noindent Note: \textbf{Show your work!} In particular, show all of the probability distributions involved in the model (namely, $P(\mathrm{class})$ and $P(\mathrm{feature|class})$) and all of the steps used to calculate them. It is recommended that you create (conditional) probability tables such as those shown below.\newpage

\begin{center}
\begin{tabular}{|c|c|}
\hline 
class & $P(\mathrm{class})$ \\ 
\hline 
comedy &  \\ 
\hline 
action &  \\ 
\hline 
\end{tabular} 
\end{center}

\begin{center}
\begin{tabular}{|c|c|c|c|}
\hline 
\multicolumn{2}{|c|}{\multirow{2}{*}{$P(\mathrm{feature|class})$}} & \multicolumn{2}{|c|}{class} \\ 
\cline{3-4}
\multicolumn{2}{|c|}{} & comedy & action \\ 
\hline 
\multirow{4}{*}{feature} & fast &  &  \\ 
\cline{2-4}
& couple &  &  \\ 
\cline{2-4}
& shoot &  &  \\ 
\cline{2-4}
& fly &  &  \\ 
\hline
\end{tabular} 
\end{center}

\section{Logistic Regression}

(You may find the discussion in Chapter 5 of the Jurafsky and Martin book helpful.)\\

\noindent Next, we will train a logistic regression classifier, using the same five training examples as before. We will use the word counts of ``fast", ``couple", ``shoot", and ``fly" as our features $x_\mathrm{fast}$, $x_\mathrm{couple}$, $x_\mathrm{shoot}$, and $x_\mathrm{fly}$, respectively, but we will not do any smoothing. Suppose also that we map the class ``comedy" to $y=1$ and the class ``action" to $y=0$. Finally, we will initialize our feature weights and bias term to 0: $w_\mathrm{fast}=w_\mathrm{couple}=w_\mathrm{shoot}=w_\mathrm{fly}=b=0$.\\

\noindent We will train our classifier using gradient descent. We divide our training data into two mini-batches: the first three examples in the first batch, and the last two in the second batch.

\begin{enumerate}
\item Process the first mini-batch.
\begin{enumerate}
\item First, compute the gradient for each example in the mini-batch.
\begin{align*}
\nabla L &= \begin{bmatrix}
\dfrac{\partial L}{\partial w_\mathrm{fast}} \\[11pt]
\dfrac{\partial L}{\partial w_\mathrm{couple}} \\[11pt]
\dfrac{\partial L}{\partial w_\mathrm{shoot}} \\[11pt]
\dfrac{\partial L}{\partial w_\mathrm{fly}} \\[11pt]
\dfrac{\partial L}{\partial b}
\end{bmatrix} = \begin{bmatrix}
\vdots \\[11pt]
\vdots \\[11pt]
\vdots \\[11pt]
\vdots \\[11pt]
\vdots
\end{bmatrix}
\end{align*}

Then, compute the overall gradient. We will follow Jurafsky and Martin, and say that the mini-batch gradient is the average of the individual gradients.
\item Update the weight vector. Use a constant learning rate $\eta=0.1$.
\begin{align*}
\theta &= \begin{bmatrix}
w_\mathrm{fast} \\[7pt]
w_\mathrm{couple} \\[7pt]
w_\mathrm{shoot} \\[7pt]
w_\mathrm{fly} \\[7pt]
b
\end{bmatrix} = \begin{bmatrix}
\vdots \\
\vdots \\
\vdots \\
\vdots \\
\vdots
\end{bmatrix}
\end{align*}

\end{enumerate}
\item Next, process the second mini-batch. Compute the gradient and update the weight vector as before, using the weights and bias you learned in step 1.
\item Suppose we stop training here (i.e., after one epoch). You are given the same document D as before:\\

\hspace{22pt}fast couple shoot fly\\

Compute the probability that D has the class ``comedy".
\end{enumerate}

\noindent Reminder: \textbf{Show your work!}

\section*{Submission Instructions}

Please submit your solutions (in PDF format) to LATTE.

\end{document}
